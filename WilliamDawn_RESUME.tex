%%%%%%%%%%%%%%%%%%%%%%%%%%%%%%%%%%%%%%%%%
% Medium Length Graduate Curriculum Vitae
% LaTeX Template
% Version 1.1 (9/12/12)
%
% This template has been downloaded from:
% http://www.LaTeXTemplates.com
%
% Original author:
% Rensselaer Polytechnic Institute (http://www.rpi.edu/dept/arc/training/latex/resumes/)
%
% Important note:
% This template requires the res.cls file to be in the same directory as the
% .tex file. The res.cls file provides the resume style used for structuring the
% document.
%
%%%%%%%%%%%%%%%%%%%%%%%%%%%%%%%%%%%%%%%%%

%----------------------------------------------------------------------------------------
%	PACKAGES AND OTHER DOCUMENT CONFIGURATIONS
%----------------------------------------------------------------------------------------

\documentclass[margin, 10pt]{res} % Use the res.cls style, the font size can be changed to 11pt or 12pt here

\renewcommand{\familydefault}{\sfdefault}
\usepackage{helvet} % Default font is the helvetica postscript font
%\usepackage{newcent} % To change the default font to the new century schoolbook postscript font uncomment this line and comment the one above

\setlength{\textwidth}{5.1in} % Text width of the document

\usepackage{amsmath}
\usepackage{amssymb}
\usepackage{microtype}
\usepackage{enumitem}
\setlist{nolistsep}
\newcommand{\backwardspace}{-8pt}

\begin{document}

%----------------------------------------------------------------------------------------
%	NAME AND ADDRESS SECTION
%----------------------------------------------------------------------------------------

\moveleft.5\hoffset\centerline{\large\bf William C. Dawn} % Your name at the top
 
\moveleft\hoffset\vbox{\hrule width\resumewidth height 1pt}\smallskip % Horizontal line after name; adjust line thickness by changing the '1pt'
 
\moveleft.5\hoffset\centerline{1422 Pitching Wedge Dr., Apt. 205} % Your address
\moveleft.5\hoffset\centerline{Raleigh, NC 27603}
\moveleft.5\hoffset\centerline{(540) 645-2576 \; $\blacklozenge$ \; wcdawn@ncsu.edu}

%----------------------------------------------------------------------------------------

\begin{resume}

%----------------------------------------------------------------------------------------
%	EDUCATION SECTION
%----------------------------------------------------------------------------------------

\vspace{-12pt}
 
\section{EDUCATION}  
\textbf{Graduate Student, Nuclear Engineering} \hfill \textit{May 2017 - Present}\\
North Carolina State University --- Raleigh, NC \hfill GPA: 3.67
\begin{itemize}
    \item Currently developing simulation suite to simulate Sodium-cooled Fast Reactors (SFRs).
    \item Solution of neutron diffusion equation via Finite Element Method (FEM) with fully coupled thermal hydraulics and thermal expansion.
    \item M.S. expected May 2019. Ph.D. expected May 2022.
\end{itemize}
\vspace{\backwardspace}
\textbf{B.S. Nuclear Engineering} \hfill \textit{August 2013 - May 2017} \\
North Carolina State Univeristy --- Raleigh, NC \hfill GPA: 4.00
\begin{itemize}
    \item Led senior design team responsible for interfacing with corporate sponsor.
    \item Designed thermal neutron test source and incorporated into GE's PRISM Sodium-cooled Fast Reactor (SFR) design.
\end{itemize}

%----------------------------------------------------------------------------------------
%	PROFESSIONAL EXPERIENCE SECTION
%----------------------------------------------------------------------------------------
 
\section{WORK \\ EXPERIENCE}

\textbf{CASL Graduate Assistant}\\
Consortium for Advanced Simulation of LWRs (CASL) \hfill \textit{August 2018}
\begin{itemize}
    \item Facilitated 
    \item Provided technical and IT support for student reactor design simulations.
    \item 
\end{itemize}

\vspace{\backwardspace}
\textbf{NESLS Engineering Intern}\\
Oak Ridge National Laboratory --- Oak Ridge, TN \hfill \textit{May 2017 - August 2017}
\begin{itemize}
    \item Added fast neutron cross section library capabilities to MPACT neutron transport code via ISOTXS file reader.
    \item Simulated fast-neutron chloride molten-salt reactor in steady-state and depletion simulations.
\end{itemize}

\vspace{\backwardspace}
\textbf{CASL Undergraduate Scholar} \\
Consortium for Advanced Simulation of LWRs (CASL) \hfill \textit{August 2015 - May 2017}
\begin{itemize}
    \item Reduced computing time by 30\% by improving steam tables in COBRA-TF.
    \item Performed code comparisons to verify simulation results in MCNP and MPACT.
\end{itemize}

\vspace{\backwardspace}
\textbf{Edison Engineering Intern} \\
GE Power --- Wilmington, NC \\
PRISM \hfill \textit{May 2016 - August 2016}
\begin{itemize}
    \item Acknowledged contributor on journal articles ``PRISM Reference Fuel Design.''
    \item Disclosed two patents relating to ESBWR and one patent related to PRISM.
\end{itemize}
\vspace{\backwardspace}
LOCA \& Containment \hfill \textit{May 2015 - August 2015}
\begin{itemize}
    \item Analyzed reactor transients using TRACG to support 10\% power uprate.
    \item Created automated data visualization and animation packages using MATLAB.
\end{itemize}

\vspace{\backwardspace}
\textbf{Licensed Reactor Operator} \\
NCSU PULSTAR Research Nuclear Reactor \hfill \textit{August 2014 - May 2017}

%----------------------------------------------------------------------------------------
%	RELEVANT SKILLS SECTION
%----------------------------------------------------------------------------------------

\section{RELEVANT \\ SKILLS}

{\sl Programming Languages:} FORTRAN, MATLAB, Python, C++, C, \LaTeX. \\
{\sl Simulation Packages:} DIF3D, MC\textsuperscript{2}-3, REBUS, MCNP, MPACT, CTF. \\
{\sl General Proficiencies:} Bash scripting, GitHub \& GitLab Project Management.

%----------------------------------------------------------------------------------------
%	AWARDS SECTION
%---------------------------------------------------------------------------------------- 
\section{AWARDS}
\textbf{Fellow}, Nuclear Engineering University Program (NEUP) \hfill \textit{May 2017 - Present} \\
\textbf{Scholarship}, American Nuclear Society \hfill \textit{May 2017 - May 2018}

\end{resume}
\end{document}